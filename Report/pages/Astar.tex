\subsection{A* Search with heuristic}
\noindent A* search is a heuristic-driven algorithm that can be effectively applied to solving CNF problems by navigating the space of partial variable assignments. In this approach, each state represents a partial assignment of truth values to the CNF variables, and the algorithm uses a heuristic to estimate the cost from the current state to a complete, satisfying solution. Specifically, the heuristic can be defined as the number of clauses that are fully assigned but remain unsatisfied. This metric guides the search by prioritizing states that are closer to satisfying all clauses, thus reducing the exploration of paths that are likely to lead to dead ends. By systematically expanding these states and pruning those that immediately violate any clause, A* efficiently converges on a complete assignment that satisfies the entire CNF formula, if one exists.

\subsubsection{Pseudocode}
\begin{algorithm}[H]
    \caption{A* Search for Hashiwokakero (\textit{grid})}
    \label{alg:astar_hashiwokakero}
    \begin{algorithmic}[1]
        \State \textbf{Input:} \textit{grid} (Hashiwokakero puzzle grid)
        \State \textbf{Output:} Solution to the puzzle or failure
        \State Encode the puzzle into CNF: \textit{cnf, edge\_vars, islands, variables} $\gets$ \textit{encode\_hashi(grid)}
        \If {no islands exist}
        \State \Return failure
        \EndIf
        \State Initialize priority queue: \textit{open\_list} $\gets$ [(initial\_state)]
        \State Initialize \textit{nodes\_expanded} $\gets$ 0
        \While {\textit{open\_list} is not empty}
        \State \textit{state} $\gets$ dequeue(\textit{open\_list})
        \State \textit{nodes\_expanded} $\gets$ \textit{nodes\_expanded} + 1
        \If {\textit{state.assignment} is complete and satisfies all clauses}
        \State \Return \textit{generate\_output(grid, islands, solution)}
        \EndIf
        \ForAll {neighbor states of \textit{state}}
        \If {neighbor is valid (no violated clauses)}
        \State Compute heuristic: \textit{h} $\gets$ \textit{compute\_heuristic(neighbor)}
        \State Compute cost: \textit{f} $\gets$ \textit{g + h}
        \State Enqueue \textit{neighbor} into \textit{open\_list}
        \EndIf
        \EndFor
        \EndWhile
        \State \Return failure
    \end{algorithmic}
\end{algorithm}

\subsubsection{Implementation}
\begin{itemize}
    \item \textbf{compute\_heuristic:} Compute the heuristic value for a given partial assignment. The heuristic is the number of clauses that are fully assigned but unsatisfied.
    \item \textbf{is\_clause\_violated:} Checks if a clause is violated by confirming that all its variables are assigned and none of its literals evaluate to True. It is used to prune branches that cannot lead to a valid solution.
    \item \textbf{is\_complete\_assignment:} Determines whether every variable in the problem has been assigned a value. It simply checks if the assignment covers all variables.
    \item \textbf{check\_full\_assignment:} Verifies that a complete assignment satisfies every clause in the CNF. It confirms the validity of the overall solution by ensuring no clause is left unsatisfied.
    \item \textbf{expand\_state:} Expands the current state by assigning the next unassigned variable with both True and False, generating new partial assignments. It prunes any branch immediately if a clause is violated by the new assignment.
    \item \textbf{solve\_with\_astar:} Encodes the Hashi puzzle into a CNF and employs an A* search to explore the space of assignments. It validates complete solutions and generates the final puzzle output upon finding a valid assignment.
\end{itemize}


\subsubsection{Heuristics in A* Algorithm}
The code implements A* search to solve a CNF-encoded Hashi puzzle. In this implementation, the total cost function is defined as:
\[
    f(n) = g(n) + h(n)
\]

where:
\begin{itemize}
    \item \( g(n) \): This is represented by the current level of the search, which corresponds to the number of variables that have been assigned values so far.

    \item \( h(n) \): The heuristic is computed by the compute\_heuristic function. It counts the number of clauses that are fully assigned yet remain unsatisfied. This estimate reflects how "far" the current partial assignment is from satisfying the overall CNF.
\end{itemize}


The algorithm starts with an empty assignment and iteratively expands states by assigning the next variable (selected in sorted order). It prunes any branch where a clause is already violated to avoid unnecessary exploration. Each new state is pushed into a priority queue (implemented with a heap) based on its \( f(n) \) value, ensuring that states estimated to be closer to a solution are explored first.

Once a complete assignment is reached, the algorithm verifies that it satisfies all clauses, extracts a model, validates it against the puzzle's constraints, and finally generates the corresponding output if the solution is correct. Overall, the design carefully integrates the A* components \( g(n) \), \( h(n) \), and state expansion—to efficiently search for a valid solution.

\subsubsection{Time and Space Complexity}
\textbf{Time Complexity:} \( O(b^d) \) in the worst case, where b is the branching factor and d is the solution depth. However, a well-designed heuristic significantly reduces the search space. If the heuristic is admissible and consistent, A* efficiently finds an optimal solution.

\textbf{Space Complexity:} \( O(b^d) \), as the algorithm stores all generated nodes in memory. This can be a limiting factor for large problem instances but ensures completeness and optimality.



\subsection*{Strengths}
\begin{itemize}
    \item \textbf{Efficient Search:} A* narrows down the search space using a heuristic function, prioritizing more promising configurations. This makes it much faster than brute-force search, particularly for puzzles of moderate size.
    \item \textbf{Optimality:} When using an admissible heuristic (one that does not overestimate the cost), A* guarantees that the solution found will be optimal, providing the best possible configuration.
    \item \textbf{Scalability:} The use of heuristics helps A* scale better than brute-force algorithms, as it avoids exploring irrelevant or unlikely configurations. It can handle puzzles of increasing size more efficiently.
    \item \textbf{Flexibility:} The heuristic can be adapted to different puzzle constraints, allowing for more customization based on specific requirements of the puzzle or variations of Hashiwokakero.
\end{itemize}

\subsection*{Limitations}
\begin{itemize}
    \item \textbf{Complex Heuristic Design:} Designing an effective heuristic that accurately reflects the puzzle's complexity and constraints can be challenging. A poorly designed heuristic may lead to inefficient searches or suboptimal solutions.
    \item \textbf{Memory Usage:} A* requires storing a large number of states in memory as it explores the search space. This can lead to high memory consumption, especially for larger puzzles with many possible configurations.
    \item \textbf{Computational Cost:} Despite its more efficient search compared to brute-force, A* can still be computationally expensive, particularly for large puzzles. The evaluation of multiple potential configurations and heuristic calculations can lead to high computational overhead.
    \item \textbf{Dependence on Heuristic Quality:} The effectiveness of A* is directly tied to the quality of the heuristic function. If the heuristic is poorly chosen, the algorithm may perform poorly or even fail to find an optimal solution within a reasonable time frame.
\end{itemize}

\subsection*{Conclusion}
The A* algorithm, when applied to solving the Hashiwokakero puzzle using a well-designed heuristic, provides a significantly more efficient and scalable approach compared to brute-force methods. By utilizing heuristics such as remaining bridges and constraint satisfaction, A* focuses the search on promising configurations, reducing unnecessary exploration of infeasible states. The algorithm guarantees optimality if an admissible heuristic is used and effectively handles the constraints of the puzzle. However, its success depends heavily on the quality of the heuristic and the complexity of the puzzle. While it offers a powerful solution for medium-sized puzzles, larger instances may still pose challenges due to high memory and computation costs.

